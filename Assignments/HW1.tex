\documentclass[en,hazy,blue,12pt,normal]{elegantnote}

\usepackage{array}

\usepackage{amsmath, amssymb, bm, color, framed, graphicx, hyperref, mathrsfs, fontspec, geometry, extarrows, amsthm}

\DeclareMathOperator{\e}{\!\!\;\mathrm e}
\DeclareMathOperator{\Cov}{Cov}
\DeclareMathOperator{\Var}{Var}
\DeclareMathOperator{\var}{var}
\DeclareMathOperator{\tr}{tr}
\DeclareMathOperator{\diag}{diag}
\newcommand{\p}{\partial}
\renewcommand{\d}{\mathop{}\!\mathrm{d}}
\newcommand{\MR}{\mathbb R}
\newcommand{\MC}{\mathbb C}
\newcommand{\MF}{\mathbb F}
\newcommand{\MZ}{\mathbb Z}
\newcommand{\MN}{\mathbb N}
\newcommand{\MCF}{\mathscr F}
\renewcommand{\Re}{\operatorname{Re}}
\renewcommand{\Im}{\operatorname{Im}}
\renewcommand{\boldsymbol}{\bm}
\renewcommand{\i}{\mathrm i}

\DeclareMathOperator{\Arg}{Arg}
\DeclareMathOperator{\I}{I}
\usepackage{tkz-euclide}
\numberwithin{equation}{section}
\numberwithin{subsection}{section}

\lstset{
    language=R,
    basicstyle=\ttfamily,
    keywordstyle=\color{blue},
    commentstyle=\color{gray},
    frame=single,
    breaklines=true
}


\title{DSC291 Stochastic Optimization\\Problem Set 1}
\author{Zeyu Bian}
\date{\today}

\begin{document}

\maketitle

\section{Theory}

\begin{homework}
    Given a positive definite matrix $A \in \mathbf{R}^{d \times d}$, show that the assignment $\langle v, w\rangle_A:=\langle A v, w\rangle$ is an inner product on $\mathbf{E}$, with the induced norm $\|v\|_{\mathcal{A}}=\sqrt{\langle A v, v\rangle}$.
\end{homework}

\begin{proof}
    We verify the definition of inner product on $\mathbf{E}$:

\textbf{1) Bilinearity}

Because the map $v \mapsto A v$ is linear and the Euclidean inner product is bilinear:

\begin{itemize}
    \item  Linearity in the first argument: for $\alpha, \beta \in \mathbb{R}$ and $v_1, v_2, w \in \mathbf{E}$,
        \begin{align}
            \left\langle\alpha v_1+\beta v_2, w\right\rangle_A&=\left\langle A\left(\alpha v_1+\beta v_2\right), w\right\rangle=\left\langle\alpha A v_1+\beta A v_2, w\right\rangle \\
            &=\alpha\left\langle A v_1, w\right\rangle+\beta\left\langle A v_2, w\right\rangle=\alpha\left\langle v_1, w\right\rangle_A+\beta\left\langle v_2, w\right\rangle_A .
        \end{align}
    \item Linearity in the second argument: for $\alpha, \beta \in \mathbb{R}$ and $v, w_1, w_2 \in \mathbf{E}$,
        \begin{align}
            \left\langle v, \alpha w_1+\beta w_2\right\rangle_A=\left\langle A v, \alpha w_1+\beta w_2\right\rangle=\alpha\left\langle A v, w_1\right\rangle+\beta\left\langle A v, w_2\right\rangle=\alpha\left\langle v, w_1\right\rangle_A+\beta\left\langle v, w_2\right\rangle_A .
        \end{align}
\end{itemize}

So $\langle\cdot, \cdot\rangle_A$ is bilinear.

\textbf{2) Symmetry}

$$
\langle v, w\rangle_A=(A v)^{\top} w=v^{\top} A^{\top} w=v^{\top} A w .
$$


Similarly,

$$
\langle w, v\rangle_A=w^{\top} A v
$$


But since $A$ is symmetric, $v^{\top} A w=w^{\top} A v$, hence

$$
\langle v, w\rangle_A=\langle w, v\rangle_A
$$

\textbf{3) Positive definiteness}

For any $v \in \mathbf{E}$,

$$
\langle v, v\rangle_A=\langle A v, v\rangle=v^{\top} A v .
$$


Because $A$ is positive definite, $v^{\top} A v>0$ for all $v \neq 0$, and clearly $0^{\top} A 0=0$. Therefore,

$$
\langle v, v\rangle_A \geq 0, \quad \text { and } \quad\langle v, v\rangle_A=0 \Longleftrightarrow v=0 .
$$


All axioms hold, so $\langle\cdot, \cdot\rangle_A$ is an inner product on $\mathbf{E}$.

\textbf{Induced norm}

The norm induced by an inner product is

$$
\|v\|_A:=\sqrt{\langle v, v\rangle_A} .
$$


Here,

$$
\|v\|_A=\sqrt{\langle v, v\rangle_A}=\sqrt{\langle A v, v\rangle}=\sqrt{v^{\top} A v},
$$
\end{proof}

\begin{homework}
    Define the function

$$
f(x)=\frac{1}{2}\langle\mathcal{A} x, x\rangle+\langle v, x\rangle+c
$$

where $\mathcal{A}: \mathbf{E} \rightarrow \mathbf{E}$ is a linear operator, $v$ lies in $\mathbf{E}$, and $c$ is a real number. Suppose that $\mathcal{A}$ is self-adjoint in the sense that it satisfies $\langle\mathcal{A} x, y\rangle=\langle x, \mathcal{A} y\rangle$ for all $x, y \in \mathbf{E}$. Derive the equation:

$$
\nabla f(x)=\mathcal{A} x+v \quad \text { and } \quad \nabla^2 f(x)=\mathcal{A} .
$$
\end{homework}

\begin{proof}
    We compute 
    \[
\begin{aligned}
f(x+h) & =\frac{1}{2}\langle\mathcal{A}(x+h), x+h\rangle+\langle v, x+h\rangle+c \\
& =\frac{1}{2}\langle\mathcal{A} x+\mathcal{A} h, x+h\rangle+\langle v, x\rangle+\langle v, h\rangle+c \\
& =\frac{1}{2}(\langle\mathcal{A} x, x\rangle+\langle\mathcal{A} x, h\rangle+\langle\mathcal{A} h, x\rangle+\langle\mathcal{A} h, h\rangle)+\langle v, x\rangle+\langle v, h\rangle+c .
\end{aligned}
\]

Then
\[f(x+h)-f(x)=\frac{1}{2}(\langle\mathcal{A} x, h\rangle+\langle\mathcal{A} h, x\rangle+\langle\mathcal{A} h, h\rangle)+\langle v, h\rangle =\langle\mathcal{A} x, h\rangle+\langle v, h\rangle+\frac{1}{2}\langle\mathcal{A} h, h\rangle .\]

By definition, the gradient is the unique vector such that
\[D f(x)[h]=\langle\nabla f(x), h\rangle \quad \text { for all } h \in \mathbf{E} .\]

Therefore from the expansion we have
\[\nabla f(x)=\mathcal{A} x+v .\]

\textbf{Hessian Computation:}

Since $\mathcal{A}$ is linear and $v$ is constant,

$$
D(\nabla f)(x)[h]=\mathcal{A} h .
$$

Therefore
\[\nabla^2 f(x)=\mathcal{A}.\]
\end{proof}

\begin{homework}
    Consider a function $f: U \rightarrow \mathbf{R}$ and a linear mapping $\mathcal{A}: \mathbf{Y} \rightarrow \mathbf{E}$ and define the composition $h(x)=f(\mathcal{A} x)$.
    
1. Show that if $f$ is differentiable at $\mathcal{A} x$, then

$$
\nabla h(x)=\mathcal{A}^* \nabla f(\mathcal{A} x)
$$

2. Show that if $f$ is twice differentiable at $\mathcal{A} x$, then

$$
\nabla^2 h(x)=\mathcal{A}^* \nabla^2 f(\mathcal{A} x) \mathcal{A}
$$
\end{homework}

\begin{proof}
    1. By the chain rule
    \[D h(x)=D f(\mathcal{A} x) \circ \mathcal{A},\]

    By definition of gradient (Riesz representation),

$$
D f(z)[w]=\langle\nabla f(z), w\rangle_{\mathbf{E}} \quad \text { for all } w \in \mathbf{E} .
$$


So with $w=\mathcal{A v}$,

$$
D h(x)[v]=\langle\nabla f(\mathcal{A} x), \mathcal{A} v\rangle_{\mathbf{E}} .
$$

Now we use the adjoint property to move $\mathcal{A}$ to the other side:

$$
\langle\nabla f(\mathcal{A} x), \mathcal{A} v\rangle_{\mathbf{E}}=\left\langle\mathcal{A}^* \nabla f(\mathcal{A} x), v\right\rangle_{\mathbf{Y}} .
$$


But also, by definition of $\nabla h(x)$,

$$
D h(x)[v]=\langle\nabla h(x), v\rangle_{\mathbf{Y}} \quad \text { for all } v \in \mathbf{Y} .
$$


Therefore, for all $v$,

$$
\langle\nabla h(x), v\rangle_{\mathbf{Y}}=\left\langle\mathcal{A}^* \nabla f(\mathcal{A} x), v\right\rangle_{\mathbf{Y}} .
$$


By uniqueness of the Riesz representative, we conclude

$$
\nabla h(x)=\mathcal{A}^* \nabla f(\mathcal{A} x) .
$$

2. By the chain rule we have

\[D^2 h(x)[u, v]=D^2 f(\mathcal{A} x)[\mathcal{A} u, \mathcal{A} v] .\]

$$
D^2 f(z)[p, q]=\left\langle\nabla^2 f(z) p, q\right\rangle_{\mathbf{E}} \quad \text { for all } p, q \in \mathbf{E},
$$

and similarly for $h$ :

$$
D^2 h(x)[u, v]=\left\langle\nabla^2 h(x) u, v\right\rangle_{\mathbf{Y}} \quad \text { for all } u, v \in \mathbf{Y} .
$$

Now we compute:

$$
\begin{aligned}
D^2 h(x)[u, v] & =D^2 f(\mathcal{A} x)[\mathcal{A} u, \mathcal{A} v] \\
& =\left\langle\nabla^2 f(\mathcal{A} x)(\mathcal{A} u), \mathcal{A} v\right\rangle_{\mathbf{E}} \\
& =\left\langle\mathcal{A}^*\left(\nabla^2 f(\mathcal{A} x) \mathcal{A} u\right), v\right\rangle_{\mathbf{Y}}.
\end{aligned}
$$


Thus, for all $u, v$,

$$
\left\langle\nabla^2 h(x) u, v\right\rangle_{\mathbf{Y}}=\left\langle\mathcal{A}^* \nabla^2 f(\mathcal{A} x) \mathcal{A} u, v\right\rangle_{\mathbf{Y}}
$$


By uniqueness again, the operators must match:

$$
\nabla^2 h(x)=\mathcal{A}^* \nabla^2 f(\mathcal{A} x) \mathcal{A} .
$$
\end{proof}

\begin{homework}
    Define the two sets

$$
\begin{aligned}
\mathbf{R}_{++}^n & :=\left\{x \in \mathbf{R}^n: x_i>0 \text { for all } i=1, \ldots, n\right\}, \\
\mathbf{S}_{++}^n & :=\left\{X \in \mathbf{S}^n: X \succ 0\right\} .
\end{aligned}
$$


Consider the two functions $f: \mathbf{R}_{++}^n \rightarrow \mathbf{R}$ and $F: \mathbf{S}_{++}^n \rightarrow \mathbf{R}$ given by

$$
f(x)=-\sum_{i=1}^n \log x_i \quad \text { and } \quad F(X)=-\ln \operatorname{det}(X),
$$

respectively. Note, from basic properties of the determinant, the equality $F(X)=f(\lambda(X))$, where we set $\lambda(X):=\left(\lambda_1(X), \ldots, \lambda_n(X)\right)$.

1. Find the derivatives $\nabla f(x)$ and $\nabla^2 f(x)$ for $x \in \mathbf{R}_{++}^n$.

2. Using the property $\operatorname{tr}(A B)=\operatorname{tr}(B A)$, prove $\nabla F(X)=-X^{-1}$ and $\nabla^2 F(X)[V]=X^{-1} V X^{-1}$ for any $X \succ 0$.

3. Show

$$
\left\langle\nabla^2 F(X)[V], V\right\rangle=\left\|X^{-\frac{1}{2}} V X^{-\frac{1}{2}}\right\|_F^2
$$

for any $X \succ 0$ and $V \in \mathcal{S}^n$. Deduce that the operator $\nabla^2 F(X): \mathbf{S}^n \rightarrow \mathbf{S}^n$ is positive definite.
\end{homework}

\begin{proof}
    1.  We directly compute
    \begin{align}
        \nabla f(x) = -(\frac{1}{x_1},\frac{1}{x_2},\dots,\frac{1}{x_n})^{\top},\quad \nabla^2 f(x) = \operatorname{diag}(\frac{1}{x_1^2},\frac{1}{x_2^2},\dots,\frac{1}{x_n^2}).
    \end{align}

    2. We compute the differential form:

    \[\nabla F(X) = \frac{1}{|X|} \frac{\partial|X|}{\partial X}=\frac{1}{|X|}\left(X^*\right)^\top=\left(-X^{-1}\right)^\top=-X^{-1},\]
where $X^*$ is the adjugate matrix and we use the Leibniz (permutation) formula

$$
\operatorname{det}(X)=\sum_{\pi \in S_n} \operatorname{sgn}(\pi) \prod_{i=1}^n x_{i, \pi(i)},
$$

For second order directional derivative, we differentiate two sides of the following equality:
\[X^{-1}X = I,\]

to get
\[d(X)X^{-1} + Xd(X^{-1}) = 0.\]

Therefore
\[d(X^{-1})[V] = -X^{-1}VX^{-1}.\]

So
\[\nabla^2F(X)[V] = -\nabla (X^{-1})[V] = X^{-1}VX^{-1}, \quad \forall X\succ 0.\]

3.
\begin{align}
    \left\langle\nabla^2 F(X)[V], V\right\rangle &= \tr (X^{-1}V^\top X^{-1}V)= \tr\left((X^{-\frac{1}{2}}V^\top X^{-\frac{1}{2}})(X^{-\frac{1}{2}}VX^{-\frac{1}{2}})\right)\\
    &=\tr ((X^{-\frac{1}{2}}VX^{-\frac{1}{2}})^\top(X^{-\frac{1}{2}}VX^{-\frac{1}{2}}))\\
    &= \|X^{-\frac{1}{2}}VX^{-\frac{1}{2}}\|_F^2.
\end{align}
\end{proof}

\section{Computation}
\subsection{Background and Objective}

In this assignment, you will study and compare the empirical behavior of gradient descent (GD) and stochastic gradient descent (SGD) on a small-scale regression task. The goal is to understand how step size, stochasticity, iterate averaging, and mini-batch size affect optimization speed, stability, and generalization. You will work with the diabetes dataset, a standard benchmark in statistical learning, where the task is to predict disease progression one year after baseline. This assignment combines implementation, experimentation, and theoretical interpretation.


\subsection{Dataset}

The dataset consists of:
\begin{itemize}
    \item $n=442$ patient samples,
    \item $d=10$ real-valued features per patient, i.e. data vectors $x_i \in \mathbf{R}^{10}$ for $i=1, \ldots, n$
    \item a target variable representing disease progression, i.e. the label $y_i \in \mathbf{R}$ for $i=1, \ldots, n$.
\end{itemize}

The data are already normalized and centered.
Data access. You should load the data using scikit-learn. In Python, this can be done as follows:

from sklearn.datasets import load\_diabetes

X, y = load\_diabetes(return\_X\_y=True)

You should randomly split the data into:
\begin{itemize}
    \item $75 \%$ training set (approximately 330 samples),
    \item $25 \%$ validation set (approximately 110 samples).
\end{itemize}

Fix a random seed to ensure reproducibility and report it in your submitted work.

\subsection{Problem Setup}

We consider regularized linear regression. Given training data $\left\{\left(x_i, y_i\right)\right\}_{i=1}^n$ with $x_i \in \mathbb{R}^d$ and $y_i \in \mathbb{R}$, define the objective

\begin{align}\label{eq. 2.3 loss}
    f(w)=\frac{1}{n} \sum_{i=1}^n \frac{1}{2}\left(x_i^{\top} w-y_i\right)^2+\frac{\lambda}{2}\|w\|^2,
\end{align}
where $\lambda \geq 0$ is a regularization parameter.

Throughout the assignment, you may fix $\lambda$ to a small positive value (e.g. $\lambda=10^{-3}$ ), unless otherwise stated.

\subsection{Part I: Batch Gradient Descent}

\noindent \textbf{(a) Gradient and Smoothness}

1. Derive the gradient $\nabla f(w)$ of the objective in Eq. \eqref{eq. 2.3 loss}.

2. Show that $f$ is convex with $L$-Lipscitz gradient. Express $L$ in terms of the data matrix $X$ and $\lambda$.

\begin{proof}[\solutionname]
    1. Suppose $X = (x_1,x_2,\dots, x_n)^\top \in \MR^{n\times d}, Y = (y_1,y_2,\dots,y_n)^\top \in \MR^{n}$ 
    
    Then we have
    \[f(\omega) = \frac{1}{2n}\|X\omega - Y\|_2^2 + \frac{\lambda}{2}\|\omega\|_2^2.\]
    
    We directly compute
    \[\nabla f(\omega)=\frac{1}{n} (X^\top X \omega- X^\top Y) +\lambda w.\]

    2.We compute the Hessian
    \[\nabla^2f(\omega) = \frac{1}{n}X^\top X +\lambda I \succeq 0.\]

    Therefore f is a convex function.

    We compute 
    \[\|\nabla f(w_1) - \nabla f(w_2)\|_2 \le \|\nabla^2 f(w^*)\|_{op} \|\omega_1-\omega_2\|_2\le (\frac{1}{n}\lambda_{max}(X^\top X) + \lambda)\|\omega_1-\omega_2\|_2.\]

    Therefore by definition we can choose
    \[L=\frac{1}{n}\lambda_{max}(X^\top X) + \lambda = \frac{1}{n}\|X\|_{op}^2 +\lambda.\]
\end{proof}

\noindent \textbf{(b) Constant Step-Size GD}

Consider batch gradient descent with constant step size $\eta$ :

\begin{align}\label{eq. Iterates}
w_{k+1}=w_k-\eta \nabla f\left(w_k\right) .    
\end{align}


1. Implement GD and run it for a fixed number of epochs. Experiment with step sizes

$$
\eta \in\left\{\frac{0.1}{L}, \frac{0.5}{L}, \frac{1}{L}, \frac{1.5}{L}\right\} .
$$


Plot \eqref{eq. Iterates} training loss vs. iteration and \eqref{eq. Iterates} validation loss vs. iteration.

\noindent \textbf{Questions.}
\begin{itemize}
    \item How does the convergence rate depend on $\eta$ ?
    \item What happens when $\eta>1 / L$ ?
\end{itemize}

\begin{proof}[\solutionname]

\begin{figure}[!htbp]
    \centering
    \includegraphics[width=0.8\linewidth]{Figures/hw1/Batch_GD__training_loss_vs_iteration.png}
\end{figure}

\begin{figure}[!htbp]
    \centering
    \includegraphics[width=0.8\linewidth]{Figures/hw1/Batch_GD__validation_loss_vs_iteration.png}
\end{figure}
    Let 
\[
f(w)=\frac{1}{2n}\|Xw-Y\|_2^2+\frac{\lambda}{2}\|w\|_2^2,\qquad 
\nabla f(w)=\frac{1}{n}X^\top(Xw-Y)+\lambda w,
\]
and define the (constant) Hessian
\[
H:=\nabla^2 f(w)=\frac{1}{n}X^\top X+\lambda I.
\]
Then $H\succeq 0$ and $f$ is $L$-smooth with 
\[
L=\lambda_{\max}(H)=\frac{1}{n}\lambda_{\max}(X^\top X)+\lambda.
\]
If $\lambda>0$ (ridge), then $f$ is also $\mu$-strongly convex with
\[
\mu=\lambda_{\min}(H)=\frac{1}{n}\lambda_{\min}(X^\top X)+\lambda \ \ge \ \lambda.
\]

\medskip
\noindent\textbf{Convergence rate vs.\ $\eta$.}
Let $w^\star=\arg\min f(w)$, which satisfies $Hw^\star=\frac{1}{n}X^\top Y$.
Using the GD update $w_{k+1}=w_k-\eta\nabla f(w_k)$ and $\nabla f(w^\star)=0$, we get
\[
w_{k+1}-w^\star = w_k-w^\star-\eta(\nabla f(w_k)-\nabla f(w^\star))
= (I-\eta H)(w_k-w^\star).
\]
Hence
\[
w_k-w^\star = (I-\eta H)^k (w_0-w^\star),
\]
so the linear contraction factor is
\[
\rho(\eta)=\max_i |1-\eta \lambda_i(H)|.
\]
Thus:
\begin{itemize}
\item If $\eta$ is very small, $\rho(\eta)\approx 1$ and convergence is slow.
\item As $\eta$ increases (while remaining stable), $\rho(\eta)$ decreases and convergence accelerates.
\end{itemize}

\medskip
\noindent\textbf{What happens when $\eta>1/L$?}
For general $L$-smooth convex functions, the standard ``safe'' choice is $\eta\le 1/L$; beyond this, the usual monotone-descent guarantee no longer holds in the simplest analysis.
For this \emph{quadratic} objective, GD is stable whenever
\[
0<\eta<\frac{2}{L},
\]
since then $\rho(\eta)<1$. However, when $\eta>1/L$, some eigen-directions satisfy $1-\eta\lambda_i(H)<0$, so the iterates can \emph{overshoot} and the loss may become \emph{non-monotone} (oscillatory), even though it can still converge if $\eta<2/L$.
If $\eta\ge 2/L$, then $\rho(\eta)\ge 1$ and GD diverges (the loss typically blows up).
\end{proof}

\subsection{Part II: Stochastic Gradient Descent}

\noindent \textbf{(a) Constant Step-Size SGD}

Let $\mathcal{B}_k$ be a mini-batch of size $b$ sampled uniformly from the training set. Define the SGD update:

\begin{align}\label{eq. SGD Iterates}
w_{k+1}=w_k-\eta \frac{1}{b} \sum_{i \in \mathcal{B}_k} \nabla \ell_i\left(w_k\right),
\end{align}
where

$$
\ell_i(w)=\frac{1}{2}\left(x_i^{\top} w-y_i\right)^2+\frac{\lambda}{2}\|w\|^2 .
$$

\noindent 1. Implement SGD with constant step sizes (experiment with a few stepsizes here).

\noindent 2. Compare batch sizes

$$
b \in\{1,5,20,100, n\} .
$$

\noindent 3. Plot training and validation loss as a function of the number of gradient evaluations.

\noindent \textbf{Questions.}
\begin{itemize}
    \item How does gradient noise depend on batch size?
    \item Which batch size gives the fastest decrease in validation error per gradient evaluation?
\end{itemize}

\begin{proof}[\solutionname]
\begin{figure}[!htbp]
    \centering
    \includegraphics[width=0.8\linewidth]{Figures/hw1/SGD__eta_1_00e_01___train_loss_vs_grad_evals.png}
\end{figure}

\begin{figure}[!htbp]
    \centering
    \includegraphics[width=0.8\linewidth]{Figures/hw1/SGD__eta_1_00e_01___val_loss_vs_grad_evals.png}
\end{figure}

\begin{figure}[!htbp]
    \centering
    \includegraphics[width=0.8\linewidth]{Figures/hw1/SGD__eta_5_02e_01___train_loss_vs_grad_evals.png}
\end{figure}

\begin{figure}[!htbp]
    \centering
    \includegraphics[width=0.8\linewidth]{Figures/hw1/SGD__eta_5_02e_01___val_loss_vs_grad_evals.png}
\end{figure}

\begin{figure}[!htbp]
    \centering
    \includegraphics[width=0.8\linewidth]{Figures/hw1/SGD__eta_1_00e_02___train_loss_vs_grad_evals.png}
\end{figure}

\begin{figure}[!htbp]
    \centering
    \includegraphics[width=0.8\linewidth]{Figures/hw1/SGD__eta_1_00e_02___val_loss_vs_grad_evals.png}
\end{figure}

    Recall $\ell_i(w)=\frac{1}{2}(x_i^\top w-y_i)^2+\frac{\lambda}{2}\|w\|_2^2$ and the mini-batch gradient
\[
g_k \;:=\; \frac{1}{b}\sum_{i\in\mathcal B_k}\nabla \ell_i(w_k),
\qquad w_{k+1}=w_k-\eta g_k.
\]
Because $\mathcal B_k$ is sampled uniformly, $g_k$ is an (approximately) unbiased estimator of the full gradient:
\[
\mathbb E[g_k \mid w_k] = \nabla f(w_k).
\]

\medskip
\noindent\textbf{How does gradient noise depend on batch size?}
Let $\xi_i(w):=\nabla \ell_i(w)-\nabla f(w)$ be the per-sample gradient noise (zero mean).
For (approximately) independent samples, the mini-batch noise is
\[
g_k-\nabla f(w_k)=\frac{1}{b}\sum_{i\in\mathcal B_k}\xi_i(w_k),
\]
so its covariance scales like
\[
\mathrm{Var}(g_k \mid w_k)\approx \frac{1}{b}\,\mathrm{Var}(\nabla \ell_i(w_k)).
\]
Equivalently, the \emph{standard deviation} (typical noise magnitude) decreases like $1/\sqrt{b}$.
Thus, small batches produce noisier (more variable) updates, while large batches produce more stable updates.

\medskip
\noindent\textbf{Which batch size gives the fastest decrease in validation error per gradient evaluation?}
Each SGD update with batch size $b$ uses $b$ gradient evaluations. Under a fixed computational budget measured in \#gradient evaluations, there is a trade-off:
\begin{itemize}
\item Small $b$ gives \emph{more parameter updates} per budget, which often improves initial progress.
\item Large $b$ reduces variance, which helps when close to the optimum, but it yields fewer updates per budget.
\end{itemize}

\end{proof}

\noindent \textbf{(b) SGD with Iterate Averaging}

Define the averaged iterate (Polyak-Ruppert averaging):

\begin{align}\label{eq. SGD average}
    \bar{w}_T=\frac{1}{T} \sum_{k=1}^T w_k .    
\end{align}


\noindent 1. Implement SGD with iterate averaging.

\noindent  2. Compare the performance of:
 
\begin{itemize}
    \item the final iterate $w_T$,
    \item the averaged iterate $\bar{w}_T$.
\end{itemize}

\noindent \textbf{Questions.}
\begin{itemize}
    \item How does averaging affect stability?
    \item Does averaging reduce variance in the objective value?
    \item Compare averaged SGD to batch GD under a similar computational budget.
\end{itemize}

\begin{proof}[\solutionname]
\begin{figure}[!htbp]
    \centering
    \includegraphics[width=0.8\linewidth]{Figures/hw1/SGD_b_20__eta_5_02e_01__train_loss.png}
\end{figure}

\begin{figure}[!htbp]
    \centering
    \includegraphics[width=0.8\linewidth]{Figures/hw1/SGD_b_20__eta_5_02e_01__val_loss.png}
\end{figure}
Polyak--Ruppert averaging forms
\[
\bar w_T=\frac{1}{T}\sum_{k=1}^T w_k.
\]

\medskip
\noindent\textbf{How does averaging affect stability?}
With constant step size, SGD does not converge exactly to $w^\star$ but instead wanders in a neighborhood due to gradient noise.
Averaging filters out this high-frequency noise: $\bar w_T$ changes more smoothly than $w_T$ and is typically much more stable across iterations and across random seeds.

\medskip
\noindent\textbf{Does averaging reduce variance in the objective value?}
Yes. Since $\bar w_T$ aggregates many noisy iterates, its variance is reduced relative to the last iterate.
In practice this appears as:
\begin{itemize}
\item less oscillation in training/validation loss curves,
\item lower variance of $f(\bar w_T)$ across runs,
\item often a better final validation loss under the same budget.
\end{itemize}

\medskip
\noindent\textbf{Averaged SGD vs batch GD under a similar computational budget.}
For a budget of $B$ gradient evaluations:
\begin{itemize}
\item Batch GD uses $n$ evaluations per step, i.e.\ only $B/n$ updates; it is deterministic and (for strongly convex quadratics) converges linearly in the number of updates.
\item SGD uses $b$ evaluations per step, i.e.\ $B/b$ updates; it makes many more updates per budget but suffers from noise.
\end{itemize}
Averaging typically makes SGD competitive with (and often better than) full-batch GD in terms of validation loss per gradient evaluation, especially when $n$ is large, because it combines frequent updates with reduced variance.
\end{proof}


\subsection{Part III: Effect of Batch Size}
\noindent \textbf{(a) Optimization vs. Statistical Error}

For different batch sizes:

1. Plot training and validation loss as a function of epochs.

2. Identify regimes dominated by optimization error versus statistical noise.

\begin{proof}[\solutionname]
We run SGD for a fixed number of epochs with different batch sizes $b \in \{1, 5, 20, 100, n\}$ and plot training and validation loss vs.\ epochs (see notebook: Part III(a) figures).

\textbf{Regimes:}
\begin{itemize}
    \item \textbf{Small $b$ (e.g.\ $b=1$):} High variance in stochastic gradients; the loss curve is noisier. Early on, \emph{optimization error} dominates (we are far from the optimum). As epochs increase, loss decreases but \emph{statistical noise} is visible (fluctuations).
    \item \textbf{Large $b$ (e.g.\ full batch):} Low variance; smooth curves. \emph{Optimization error} dominates throughout (fewer, noiseless updates per epoch). Less statistical noise.
    \item In summary: early epochs / small $b$ $\Rightarrow$ more statistical noise; late epochs / large $b$ $\Rightarrow$ optimization error (bias, convergence speed) dominates.
\end{itemize}
\end{proof}

\noindent\textbf{(b) Critical Batch Size}

Fix a step size $\eta$ and vary the batch size $b$.

1. Identify a batch size beyond which performance gains saturate.

2. Relate your observations to the variance of stochastic gradients.

\begin{proof}[\solutionname]
\begin{figure}[!htbp]
    \centering
    \includegraphics[width=0.8\linewidth]{Figures/hw1/Part_III_a___Training_loss_vs_epochs__different_batch_sizes.png}
\end{figure}

\begin{figure}[!htbp]
    \centering
    \includegraphics[width=0.8\linewidth]{Figures/hw1/Part_III_a___Validation_loss_vs_epochs__different_batch_sizes.png}
\end{figure}

\begin{figure}[!htbp]
    \centering
    \includegraphics[width=0.8\linewidth]{Figures/hw1/Part_IIIb_final_loss_vs_batch_size.png}
\end{figure}
We fix step size $\eta$ and vary batch size $b$; we run SGD for a fixed number of epochs and record the final train/validation loss (see notebook: Part III(b) figure).

\textbf{Critical batch size:} Beyond some $b$ (e.g.\ around 6~8 on the diabetes dataset), increasing batch size gives diminishing returns; final loss saturates.

\textbf{Relation to variance:} The variance of the minibatch gradient scales roughly like $1/b$ under i.i.d.\ sampling. For small $b$, variance is large and many updates are ``wasted'' in noise; for large $b$, variance is small and each update is close to the full gradient. Once $b$ is large enough, variance is already low and further increasing $b$ does not improve convergence much per epoch---hence saturation of performance.
\end{proof}

\end{document}